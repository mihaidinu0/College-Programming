\documentclass{article}
%babel
\usepackage[romanian]{babel}
%
\usepackage{graphicx}%doar dacă dorim să importăm grafică
%
%titlu
\title{Exemple C6}
\author{Student\footnote{Grupa:}}
\begin{document}
%afișare titlu, cuprins, listă figuri
\maketitle
\tableofcontents%afișarea cuprinsului
\listoffigures%afișarea listei figurilor
\begin{abstract}%rezumat
Se studiază: grafică în \LaTeX\ , cadrele pentru citate \verb+quote+, \verb+quotation+, crearea de liste cu \verb+itemize+, \verb+enumerate+, \verb+description+ și crearea manuală de bibliografie numerotată.  
\end{abstract}
\section{Cadre pentru realizarea de grafică}
Un text urmat de grafică în \LaTeX\, realizată cu \verb+picture+.
%
\begin{figure}[htpb]
\centering
\begin{picture}(50,50)
\setlength{\unitlength}{1pt}%seteaza unitatea de lungime la valoarea 1 pt
\put(25,0){\dashbox{.5}(25,25)[b]{TEXT}}
%obiectul dashbox (25x25)e plasat in punctul de coordonate (25,0)
\end{picture}
\caption{Un box cu text jos.}
\end{figure}
%
\begin{figure}[htpb]
\centering
\begin{picture}(50,50)
\setlength{\unitlength}{1pt}
\put(25,5){\vector(1,2){20}}
\put(25,2.6){\line(3,-1){20}}
\end{picture}
\caption{Linie și vector.}
\end{figure}
%
\begin{figure}[htpb]
\centering
\begin{picture}(50,50)
\setlength{\unitlength}{1pt}
\put(20,0){\circle{20}}
\put(20,0){\vector(0,1){10}}
\put(40,0){\circle*{10}}
\end{picture}
\caption{Cerc cu rază și disc.}
\end{figure}
%
\section{Cadre pentru evidențierea citatelor}
Text introductiv urmat de cadrele \verb+quote+ și \verb+quotation+.
\begin{quotation}
Prima lege a lui Kirchhoff (sau legea nodurilor) este o expresie a conservării sarcinii electrice.\par
Suma algebrică a intensităților curenților electrici care se întâlnesc într-un nod de rețea este egală cu zero.
\end{quotation}
A doua lege a lui Kirchhoff poate fi formulată astfel:
\begin{quote}
Suma algebrică a tensiunilor (electromotoare și pe elemente rezistive) dintr-un ochi de rețea este egală cu 0. 
\end{quote}
\section{Cadre pentru crearea de liste}
Text  cu  urmat de o listă compusă cu \verb+itemize+ și \verb+enumerate+.
\begin{itemize}
\item Fiecare element din listă are un bullet.
\item Listele pot fi încuibate.
\begin{enumerate}
\item Etichetele într-o listă numerotată sunt numere sau litere.\label{et}
\item O listă are cel puțin două elemente:\label{elem}
\begin{enumerate}
\item un prim element
\item un al doilea element\label{elel}
\end{enumerate}
\item \LaTeX\ permite patru niveluri de încuibare.
\end{enumerate}
\end{itemize}
\par
Regulile de etichetare sunt descrise în elementul \ref{et}, iar elementul \ref{elem} are sub-elementul \ref{elel} afișat la pagina \pageref{elel}.\par 
O listă creată cu \verb+description+ arată astfel:
\begin{description}
\item[Curs:] teorie și concepte de bază
\item[Proiect:] dezvoltarea unei aplicații
\item[Laborator:] exerciții
\end{description}
\section{Cadre și comenzi pentru generarea manuală a bibliografiei}
Urmează referințele bibliografice în forma numerotată.\par
La cursul de matematică se folosește lucrarea \cite{mat}, iar la bazele electrotehnicii se studiază legile descrise în \cite{elth}.
\begin{thebibliography}{a}
\bibitem{mat} I.Ionescu, Analiză matematică, Editura ALL, 2015.
\bibitem{elth} I. Daniel, Bazele electrotehnicii, Editura Politehnica Press, 2019.
\end{thebibliography}
\end{document}