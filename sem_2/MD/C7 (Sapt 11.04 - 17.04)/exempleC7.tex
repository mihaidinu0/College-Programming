\documentclass{article}
%babel
\usepackage[romanian]{babel}
%
\usepackage{amssymb}%pentru a folosi simboluri speciale in mod mat, \mathbb{•}
%\usepackage{amsmath}
%titlu
\title{Exemple C7}
\author{Student\footnote{anul IAC}}
%
%\sloppy% se folosește pentru a "relaxa" distanța dintre cuvinte pe rând
\begin{document}
\maketitle
\begin{abstract}
Se studiază: contori și cadre de tip teoremă: \verb+newtheorem+.  
\end{abstract}
\section{Contori}\label{cont}
\subsection{Cum gestionează \LaTeX\ contorii?}\label{intern}
În secțiunea \ref{cont} de la pagina \pageref{cont}, se deschide subsecțiunea \ref{intern}. Valoarea curentă a contorului \verb+subsection+ este \thesubsection.
\subsection{Ce facem noi?}\label{utilizator}
Contorul subsecțiunii \ref{utilizator} conservă contorul \ref{cont} al secțiunii principale. Valoarea Valoarea curentă a contorului \verb+subsection+ a devenit \thesubsection. \par
Ca urmare:
\begin{enumerate}
\item referim contorii entităților etichetate cu \verb+\label{•}+ cu \verb+\ref{•}+.
\item vizualizăm valorile curente ale contorilor cu \verb+\thectr+; exemplu: valoarea contorului \verb+enumi+ al item-ului curent este \theenumi.
\item adunăm o valoare la un contor; exemplu: cu \verb+\addtocounter{section}{1}+ valoarea contorului \verb+section+ devine \addtocounter{section}{1} \thesection.
\item setăm la o valoare un contor cu \verb+\setcounter{ctr}{num}+; exemplu: setăm la 4 valoarea curentă a \verb+section+ \setcounter{section}{4}, cu efectul \thesection; observați numărul  secțiunii următoare.
\item putem defini și noi contori, vom vedea imediat un exemplu.
\item putem folosi valoarea unui contor într-o expresie, cu \verb+\value{ctr}+.
\item putem schimba stilul numerotării; exemplu: cu \verb+\renewcommand{\theenumi}{\roman{enumi}}+, 
\renewcommand{\theenumi}{\roman{enumi}} itemul curent se numerotează  \theenumi.
\end{enumerate} 
\section{Unități de măsura pentru lungimi}
A se vedea C7.
\section{Cadre de tip teoremă}
Comenzile sunt:\par
 \verb+\newtheorem{env_name}{caption}[within]+ sau \par
 \verb+\newtheorem{env_name}[numbered_like]{caption}+. \par
Folosim contorul nou \verb+teor1+ în cadrul cu numele \verb+teor1+.
\newtheorem{teor1}{Teorema}
\newtheorem{teor2}[teor1]{Axioma}
\begin{teor1}
 Unele teoreme sunt numerotate.
\end{teor1}
Contorul \verb+teor2+ al cadrului \verb+teor2+ este numerotat ca \verb+teor1+ datorită argumentului \verb+[teor1]+ din comanda \verb+\newtheorem{teor2}[teor1]{Axioma}+, deci se va incrementa la comanda \verb+\begin{teor2}+ imediat după \verb+\end{teor1}+.
\begin{teor2}
Toți oamenii sunt muritori.
\end{teor2}
Folosim un nou contor \verb+teor3+ într-un nou cadru \verb+teor3+. Comanda \verb+\mathbb{R}+ ce afișează, în mod matematic, $\mathbb{R}$, necesită în preambul \verb+\usepackage{amsymb}+.
\newtheorem{teor3}{Teorema}
\begin{teor3}[Poincar\'{e}-Liapunov] Dacă valorile proprii ale operatorului liniar  $A:\mathbb{R}^n \rightarrow \mathbb{R} ^n$ au partea reală negativă, atunci poziția de echilibru $x=0$ a sistemului diferențial $\dot{x}=Ax$  este asimptotic stabilă.
\end{teor3}
Iată o definiție al cărei contor \verb+defin+ este interior lui \verb+section+.
\newtheorem{defin}{Definiție}[section]
\begin{defin} 
Valorile proprii ale unei matrice $A$ sunt rădăcinile polinomului caracteristic $P(\lambda)=\det(\lambda I -A)$.
\end{defin}
\end{document}